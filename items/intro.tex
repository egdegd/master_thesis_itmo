\specialsection{Введение}

На практике популярным подходом для решения сложных комбинаторных задач является кодирование их в конъюнктивную нормальную форму (КНФ) и вызывать САТ-солвер. Есть две основные причины почему этот подход хорошо работает для многих трудных задач: современные САТ-солверы очень эффективны и многие комбинаторные задачи естественным образом выражаются в терминах КНФ. В тоже время, КНФ кодировки не единственны. Кроме того, нет такого понятия как лучшие способ перевести задачу в КНФ: разные кодировки имеют разное число дизъюнктов, разное число переменных и разную ширину дизъюнктов. Для многих реально возникающих задач (например, конфигурация продукта~\cite{ProvingConsistencyAssertions}, назначение радиочастот~\cite{RadioLinkFrequencyAssignment} или восстановление изображений с помощью компьютерной томографии~\cite{EfficientCNFEncodingOfBooleanCardinalityConstraints}) выбранный способ кодировки напрямую влияет на время работы солвера. Причина в том, что простое представление
в виде КНФ содержит много дизъюнктов для многих булевых функций. Чтобы уменьшить количество дизъюнктов,
можно использовать недетерминированные переменные. Однако
введение недетерминированных переменных вынуждает SAT-солвер делать потенциально
большее число разветвлений. Таким образом, лучшее соотношение между числом переменных и количеством дизъюнктов определяется экспериментально\cite{DBLP:series/faia/Prestwich09}. В~\cite{10.1007/978-3-540-74970-7_35} показано, что изменения в САТ-солверах может смягчить недостатки связанные с введением новых недетерминированных  переменных. Прествич~\cite{DBLP:series/faia/Prestwich09} дает обзор разных способов перевода задачи в КНФ и рассуждает о требуемых для этого свойствах как с теоретической, так и с практической точки зрения.

Две частые конструкции, встречающиеся при переводе в КНФ - это parity ($(x_1 +x_2+\dotsb+x_n) \bmod 2$) и  at-least ($x_1+\dotsb+x_n \ge k$). Последняя обычно называется пороговой функции в области схемной сложности~\cite{threshold_function} и ограничитель кардинальности в области САТ-солверов~\cite{10.1007/978-3-540-74970-7_35}.
Известным представителем конструкции at-least является функция majority ($x_1+\dotsb+x_n > n/2$).
В библиотеки \texttt{pysat}~\cite{PySat} пользователь может выбрать до $10$ разных способов кодировки конструкции at-least.
Для экспериментальных сравнений разных кодировок кардинальных ограничений смотри~\cite{Frisch2010SATEO, atmostk,  kochemazov2016comparison}.

Функции parity ($\PAR$), она же четности, и majority ($\MAJ$), она же голосования, также являются одними из самых популярных при доказательстве нижних оценок на размеры булевых схем. Например, известно много техник для доказательства того, что функции четности и голосования требуют схем константной глубины экспоненциального размера (смотри \cite[главы~11 и~12]{DBLP:books/daglib/0028687} для обзора). В тоже время, не так много чего известно о КНФ кодировках с теоретической точки зрения. Синз~\cite{DBLP:conf/cp/Sinz05} доказал нижнюю и верхнюю границу на число дизъюнктов в КНФ кодировки функции at-least:
любая КНФ кодировка имет хотя бы $n$~дизъюнктов и существует кодировка с $7n$ дизъюнктами.
Кучера, Савик, Ворел~\cite{DBLP:journals/tcs/KuceraSV19} 
доказали нижнюю оценку $2n + o(n)$на число дизъюнктов для at-most-one.

В этой работе мы покажем связь между числом числом $m$ дизъюнктов, шириной $k$ дизъюнкта, и числом $s$ недетерминированных переменных в КНФ кодировке функции четности и голосования.
Взяв $s=O(n)$, минимальное число дизъюнктов в КНФ кодировки четности находится между $3n$~и~$4n$, в то время как любая симметрическая функция может быть закодирована, используя не больше $18n$ дизъюнктов.
Для любого $s=s(n)$, минимальное $k$~такое что четность может быть закодирована как ~$k$-КНФ - это $\frac{n}{s+1}$, с точностью до умножения на константу. 
И главное, при $s=n^{\alpha}$
(где $0 \le \alpha \le 1$~---~константа)
минимально число дизъюнктов в КНФ кодировки обеих функций честности и голосования примерно равно $2^{n^{1-\alpha}}$.

Верхние оценки получены с помощью следующей стратегии: разделить входные переменные на блоки и закодировать вычисляемую функцию для каждого блока (мы покажем это формально позже).

Нижние оценки были получены благодаря тесной связи между КНФ кодировками и схемами глубины $3$. Также мы использовали Satisfiability Coding Lemma авторов Патури, Пудлак и Зейн\cite{DBLP:journals/cjtcs/PaturiPZ99}.
Это лемма позволяет доказать нижнюю оценку $2^{\sqrt n}$ на размер схемы глубины $3$, вычисляющую функцию четности. Интересным является то, что наша нижняя оценка на число дизъюнктов моментально дает нижнюю оценку $2^{\Omega(\sqrt n)}$ для схем глубины $3$, вычисляющих функцию четности. Можно ли доказать обратное следствие является открытым вопросом. Эта связь показывает дополнительную мотивация для изучения КНФ кодировок как модель вычисления булевых функций: с одной стороны,
они часто используется на практике при переводе
реальной вычислительной задачи в формат, подходящий
для SAT-солвера, с другой стороны, нижние оценки на размер КНФ-кодировок влечет нижние оценки на размер схем глубины 3.
