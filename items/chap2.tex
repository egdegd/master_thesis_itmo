\section{Нижние оценки для КНФ кодировок}
\label{section:lowerbounds}

\subsection{Connection to~Circuit Lower Bounds}

Перед тем как доказывать нижние оценки на КНФ-кодировки функции чётности и голосования, мы покажем, что доказательство сильных нижних оценок для КНФ-кодировок является сложной задачей. Действительно, Лемма~\ref{lemma:circuit2encoding} и преобразование Цейтена позволяют нам легко переводить схемы в КНФ кодировки. Используя это преобразование нижние оценки на КНФ кодировки могут быть транслированы в нижние оценки для схем.

Для функции чётности,
лучшая известная нижняя оценка на схему глубины~3 составляет~$\Omega(2^{\sqrt{n}})$~\cite{DBLP:journals/cjtcs/PaturiPZ99}.
Если дополнительно требуется, чтобы схема была формулой, т.е. чтобы у каждого элемента выходная степень была не более~1, то лучшая нижняя оценка составляет~$\Omega(2^{2\sqrt{n}})$~\cite{DBLP:journals/eccc/Hirahara17}.
Обе нижние оценки точны до полиномиальных множителей.
Для функции голосования 
существует нижняя оценка на схему глубины~3
$2^{\Omega(\sqrt{n})}$~\cite{Hastad, Hastad1995} и 

яя граница на формулу глубины~3 $2^{O(\sqrt{n\log n})}$~\cite{DBLP:journals/eccc/Hirahara17, 10.1145/800057.808717}.
Интересно, что эти нижние оценки подтверждают, что параметры Леммы~\ref{lemma:circuit2encoding}
не могут быть существенно улучшены. Действительно, подставив в КНФ кодировку для ~$\PAR_n$ с  $s=\sqrt n$ и $m=O(\sqrt n2^{\sqrt n})$
(см.~\eqref{eq:blocks}),
можно получить~$\Sigma_3$-формулу и~$\Sigma_3$-схему размера $2^{2\sqrt n}$
и $2^{\sqrt n}$ соответственно, с точностью до полиномиальных множителей. Как обсуждалось выше,
эти оценки известны как оптимальные.

Ниже (см.~\eqref{eq:sm}) мы доказываем, что для любой КНФ-кодировки~$\PAR_n$ с 
$s$~недетерминированными переменными и $m$~дизъюнктами,
$m \ge \Omega\left(\frac{s+1}{n} \cdot 2^{n/(s+1)}\right)$.
Теперь пусть $C$~будет~$\Sigma_3(t,r)$-формулой, вычисляющей~$\PAR_n$.
Лемма~\ref{lemma:circuit2encoding} гарантирует, что $\PAR_n$ может быть закодирована как КНФ размера~$r$ с $\lceil \log t \rceil$ недетерминированными переменными. Тогда,
\[\size(C) = t + r \ge t + \Omega \left( \frac{1}{n}\cdot 2^{\frac{n}{\log t + 2}} \right) \ge \frac 1n \left( t + \Omega\left( 2^{\frac{n}{\log t + 2}}\right)\right) \ge \Omega\left(\frac{2^{\sqrt{n}}}{n}\right) \, .\]
Аналогично, если $C$~является~$\Sigma_3(t,r)$-схемой, Лемма~\ref{lemma:circuit2encoding} гарантирует, что $\PAR_n$ может быть закодирована как КНФ размером~$2rt$ с $\lceil \log t \rceil$ недетерминированными переменными. Тогда,
\[\size(C) = t + r \ge t + \Omega \left( \frac{1}{2tn}\cdot 2^{\frac{n}{\log t + 2}} \right) \ge \Omega\left(\frac{2^{\sqrt{n/2}}}{n}\right) \, .\]
Таким образом, нижние оценки для КНФ-кодировок подразумевают нижние оценки
для схем глубины~3.
Заметьте, что ни для одной булевой функции из NP мы не знаем,
как доказать нижнюю оценку $2^{\omega(\sqrt{n})}$
на размер схемы глубины~3, вычисляющей её.
(Говоря, что булева функция принадлежит классу NP, мы имеем в виду, что у нас есть бесконечная последовательность функций $\{f_n\}_{n=1}^{\infty}$ такая, что язык $\bigcup_{n=1}^{\infty}f^{-1}_n(1)$ принадлежит NP.)

дизъюнкт. 

\begin{openproblem}
	\label{open:sqrtn}
	Найдите булеву функцию из NP, которая не может 
	быть вычислена схемами глубины~3 размера $2^{O(\sqrt n)}$.
\end{openproblem}

Другой сложной открытой задачей является нахождение 
булевой функции, которая не имеет схем глубины~3 
размера $2^{O(n/\log \log n)}$ с ограничением нижней входной степени $n^{\varepsilon}$ для некоторой константы $\varepsilon<1$.
Как доказал Валлиант~\cite{Valiant1977GraphTheoreticAI},
такая функция не может быть вычислена схемами с входной степенью~2, размером~$O(n)$ и глубиной~$O(\log n)$. 
Это является известной трудной открытой задачей в теории сложности схем.
Интересно, что в своем сведении Валлиант фактически 
показывает, что функция, которая может быть вычислена булевой схемой линейного размера
и логарифмической глубины, имеет нетривиальную
КНФ-кодировку.

\begin{openproblem}
	\label{open:nloglogn}
	Найдите булеву функцию из NP, которая не может 
	быть вычислена бинарными схемами глубины $O(\log n)$
	и размером~$O(n)$.
\end{openproblem}

Фактически, для схем с входной степенью два, лучшая известная нижняя
оценка составляет $3.1n$~\cite{DBLP:conf/stoc/Li022} (даже 
если ограничить глубину до~$O(\log n)$).

\begin{openproblem}
	\label{open:32n}
	Найдите булеву функцию из NP, которая не может 
	быть вычислена схемами с входной степенью два и размером $3.2n$.
\end{openproblem}


Ниже мы показываем, что эти открытые задачи можно решать с помощью подхода КНФ-кодировок.
\begin{lemma}
	\begin{enumerate}
		\item Чтобы решить Открытую задачу~\ref{open:sqrtn}, достаточно построить функцию $f \colon \{0,1\}^n \to \{0,1\}$ из NP, такую что $f$ не имеет КНФ-кодировки с 
		$s$~недетерминированными переменными и $m=O(2^{n/s})$.
		\item Чтобы решить Открытую задачу~\ref{open:nloglogn}, достаточно построить функцию $f \colon \{0,1\}^n \to \{0,1\}$ из NP, такую что $f$ не имеет КНФ-кодировки с $s=O(\frac{n}{\log \log n})$ и $m=O(\frac{n}{\log \log n}2^{n^\varepsilon})$ для любой константы $\varepsilon > 0$.
		\item Чтобы решить Открытую задачу~\ref{open:32n}, 
		достаточно построить функцию $f \colon \{0,1\}^n \to \{0,1\}$ из NP, такую что $f$ не имеет КНФ-кодировки с $s=3.2n$ и $m=13n$.
	\end{enumerate}
\end{lemma}

\begin{proof}
	\begin{enumerate}
		\item Рассмотрим~$\Sigma_3(t,r)$-схему~$C$ размера~$t + r$. Лемма~\ref{lemma:circuit2encoding} гарантирует, что $C$ можно закодировать как КНФ размера~$m = 2rt$ с $s = \lceil \log t \rceil$ недетерминированными переменными. Так как $m=O(2^{n/s})$, $r = \frac{1}{2t}O(2^{n / \log t})$.
		Следовательно,
		\[\size(C) = t + r = t +  \frac{1}{2t}\cdot O(2^{n / \log t}) = 2^{O(\sqrt{n})}\]
		(либо $t \ge 2^{\sqrt{n/2}}$, либо $\frac{1}{2t}\cdot O(2^{n/\log t}) \ge 2^{\sqrt{n/2}}$).
		
		\item Мы докажем, что любая схема размера $O(n)$ и глубины $O(\log n)$ может быть преобразована в КНФ с требуемыми параметрами. 
		Возьмем схему~$C$ глубины~$d = O(\log n)$ с $O(n)$ элементами с входной степенью~$2$. Так как у каждого элемента входная степень~$2$, количество~$R$
		проводов не превышает $O(n)$.
		
		Как доказал Валлиант~\cite{DBLP:conf/mfcs/Valiant77},
		для любого ориентированного графа глубины~$d$ (где глубина - это длина самого длинного пути в графе) с $R$~ребрами и любым целым числом $1 \le r \le \log d$, можно удалить $\frac{r}{\log d}R$ ребер так, что глубина 
		полученного графа будет не больше $d/2^r$.
		
		Для параметра~$r$, который будет указан позже, 
		применим лемму Валлианта к схеме~$C$.
		Для каждого удаленного провода мы вводим 
		недетерминированную переменную,
		и чтобы удовлетворить её значение, 
		мы добавляем не более $2^{2^{d/2^r}}$ дизъюнктов.
		Таким образом, мы получаем КНФ-кодировку с не более чем $O(\frac{n r}{\log d})$~недетерминированными переменными,
		и не более $O(\frac{n r}{\log d} 2^{2^{d / 2^r}})$~дизъюнктов. Так как $d = O(\log n)$ и, взяв $r \approx \log (1 / \varepsilon)$ (константа), мы получаем КНФ-кодировку с $O(\frac{n}{\log \log n})$ недетерминированными переменными и $O(\frac{n}{\log \log n} 2^{n^\varepsilon})$ дизъюнктами.
		
		\item Если бы $f$ имела схему с входной степенью два и размером $3.2n$,
		то, используя Наблюдение~\ref{obs:binarycircuits}, 
		её можно было бы закодировать как КНФ с $s=3.2n$ и $m=4\cdot 3.2n 
		\le 13n$.
	\end{enumerate}
\end{proof}


В заключение этого раздела отметим, что, как это обычно бывает, легко получить
неконструктивное доказательство существования булевой функции, не имеющей маленькой КНФ-кодировки.
Поэтому основная задача заключается в нахождении \emph{явной}
такой функции, где под явной обычно понимается функция
из NP (или $E^{NP}$).
Действительно, существует
\[2^{(2n+2s)m}\]
КНФ-кодировок с $n$~входными переменными, $s$~недетерминированными переменными и $m$~дизъюнктами: существует $m$~дизъюнктов, каждый из которых является подмножеством $n$ входных и $s$ недетерминированных переменных, а также их отрицаний. Так как существует $2^{2^n}$ булевых
функций, то для 
\[(2n+2s)m < 2^n,\]
существует булева функция, которая не может быть закодирована как КНФ
с $s$~недетерминированными переменными и $m$~дизъюнктами.

\subsection{Изолированные решения}\label{sec:isolated}
В данном разделе мы докажем две технические леммы, необходимые для доказательства нижних оценок.

Основное свойство функций $\PAR$ и $\MAJ$, используемое в наших доказательствах нижних оценок, заключается в том, что у них много изолированных решений. Означивание $x \in f^{-1}(1)$ называется \emph{изолированным в направлении~$i$}, если инверсия $i$-го бита $x$ меняет значение функции, то есть $x' \in f^{-1}(0)$. Мы говорим, что $x$ \emph{$d$-изолирован}, если есть $d$ таких направлений. Под $I_{f, x}$ мы обозначаем множество направлений для $x$. Если КНФ~$F$
вычисляет~$f$, то для каждого $d$-изолированного $x \in f^{-1}(1)$ и для каждого направления $i \in I_{f, x}$, $F$ должна содержать
дизъюнкт, который удовлетворяется только $x_i$. Следуя \cite{DBLP:journals/cjtcs/PaturiPZ99}, мы называем такой дизъюнкт \emph{критическим относительно~$(x,i)$}.
Под $C_{F,x,i}$ будем обозначать кратчайший критический дизъюнкт относительно $(x,i)$.
Теперь, для $d$-изолированного удовлетворяющего набора~$x$, определим его \emph{вес} относительно~$F$ как
\[w_F(x) = \sum\limits_{i \in I_{f, x}} \frac{1}{|C_{F,x,i}|} \, .\]

В~\cite{DBLP:journals/cjtcs/PaturiPZ99} авторы доказали,
что КНФ не~может удовлетворять слишком 
много означиваний большого веса. В вышеуказанной статье это было доказано для $d=n$. 
Ниже мы~покажем, что незначительные изменения 
доказательства позволяют расширить результат на~любой~$d$.

\begin{lemma}\label{lemma:isolatedweight}
	Для любого $\mu>0$ и любого целого числа $0 \le d \le n$, 
	КНФ с $n$~переменными имеет не~более $2^{n - \mu}$ $d$-изолированных выполняющих наборов веса не~менее~$\mu$.
\end{lemma}

Пусть $F(x_1, \dotsc, x_n)$ будет КНФ, вычисляющей $f \colon \{0,1\}^n \to \{0,1\}$, и $x \in f^{-1}(1)$.
Для перестановки $\sigma \in S_n$ определим 
кодировку $\Phi_\sigma \colon \{0,1\}^n \to \{0,1\}^{\le n}$ для~$x$ следующим образом.
Переставим биты~$x$ в соответствии с~$\sigma$. 
Для каждого~$i \in [n]$, удалим $i$-й бит переставленной строки, если существует критический дизъюнкт $C_{F, x, \sigma(i)}$, 
такой что 
переменная $\sigma(i)$ появляется после всех остальных переменных в~этом дизъюнкте 
в соответствии с~порядком~$\sigma$.

Напомним, что функция кодировки $\Phi \colon S \rightarrow \{0,1\}^*$ называется \emph{префиксно-свободной}, 
если $f(s_1)$ не является префиксом~$f(s_2)$ для любых $s_1 \neq s_2 \in S$.
В~\cite[Fact~1]{DBLP:journals/cjtcs/PaturiPZ99}, 
доказано, что для префиксно-свободной кодировки~$\Phi$
со средней длиной кода $l=\sum_{s \in S}\Phi(s)/|S|$, 
выполняется $|S| \le 2^l$.   
Также показано, что $\Phi_{\sigma}$ является префиксно-свободной кодировкой.

\begin{proof}[Доказательство Леммы~\ref{lemma:isolatedweight}]
	Мы покажем, что существует перестановка~$\sigma$ такая,
	что средняя длина описания при кодировании~$\Phi_\sigma$
	$d$-изолированного решения веса хотя бы $\mu$ будет не более $n - \mu$. 
	
	Возьмём случайную перестановку~$\sigma$.
	Пусть $x$~будет $d$-изолированным решением веса $w(x) \ge \mu$.
	Поскольку бит в~$x$, соответствующий 
	переменной~$i$, удаляется с вероятностью не менее $1 / |C_{(F, x, i)}|$ при построении кодировки $\Phi_{\sigma}$, 
	то ожидаемое
	число удалённых битов не менее $\sum_{i \in I_{f, x}} 1/|C_{(F, x, i)}| \ge \mu$.
	Следовательно, существует перестановка $\sigma$, такая что средняя
	(по всем изолированным решениям веса хотя бы
	$\mu$) длина описания при кодировании $\Phi_\sigma$ будет
	не более $n - \mu$. Таким образом, количество изолированных решений веса хотя бы $\mu$ будет
	не более~$2^{n - \mu}$. 
\end{proof}

Понятие изолированного решения естественным образом распространяется на КНФ кодировки.
А именно, рассмотрим функцию $f$ и $d$-изолированный набор $x \in f^{-1}(1)$.
Пусть $F(x, y)$ будет КНФ кодировкой $f$, 
и $y \in \{0, 1\}^s$ такой, что $F(x, y) = 1$.
Тогда для любого $i \in I_{f, x}$,
$F$ содержит дизъюнкт, который становится ложным, если изменить бит $x_i$.
Мы называем его критическим относительно $(x,y,i)$.

\begin{lemma}\label{lemma:nofclauses}
	Пусть $F(x_1, \dotsc, x_n, y_1, \dotsc, y_s)$ будет КНФ кодировкой $f \colon \{0,1\}^n \to \{0,1\}$ с 
	$m$ дизъюнктами.
	Пусть $d \in [n]$ и $S = \{x \in f^{-1}(1) \colon \text{$x$ является $d$-изолированным}\}$.
	Тогда для любого $0 < \varepsilon \le d\log 2 - s - 1$,
	\[ m \ge (s+1+\varepsilon) 2^{\frac{d}{s+1+\varepsilon}} (|S|2^{-n} - 2^{-1-\varepsilon}). \]
\end{lemma}
\begin{proof}
	Рассмотрим \emph{проекцию} $F$:
	\[f(x)=\bigvee_{j \in [2^s]} F_j(x) \, .\]
	
	Мы расширяем определения $C_{F,x,i}$ и $w(x)$ до КНФ с недетерминированными переменными следующим образом. Пусть $x \in f^{-1}(1)$ является $d$-изолированным с направлениями $I = \{i_1, i_2, \dots, i_d\}$.
	Пусть $j \in [2^s]$ будет наименьшим индексом, таким что $F_j(x)=1$.
	Для $i \in I$, пусть $C'_{F,x,i}=C_{F_j,x,i}$ (то есть, мы просто берём
	первую функцию $F_j$, которая выполняется набором $x$, и берём ее критический дизъюнкт относительно $(x,i)$).
	Тогда вес $w'_F(x)$ относительно $F$ определяется как $w_{F_j}(x)$:
	
	\[w'_F(x) := w_{F_j}(x) = \sum_{i \in I} \frac{1}{|C'_{(F_j,x,i)}|}= \sum_{i \in I} \frac{1}{|C'_{(F,x,i)}|} \,. \]
	Для $l \in [n]$, пусть также $N_{l,F}(x)=|\{i \in [n] \colon |C'_{F,x,i}|=l\}|$
	будет количеством критических дизъюнктов (относительно $x$) длины $l$. Очевидно,
	\begin{equation}\label{eq:weight}
		w'_F(x)=\sum_{l \in [n]}\frac{N_{l,F}(x)}{l} \, .
	\end{equation}
	
	
	Для параметра $\varepsilon$, разобьём $S \subseteq f^{-1}(1)$
	на лёгкие и тяжёлые части:
	\begin{align}
		H &= \{x \in S  \colon w'_F(x) \ge s + 1 + \varepsilon\} \, ,\\
		L &= \{x \in S  \colon w'_F(x) < s + 1 + \varepsilon\} \, .
	\end{align}
	
	Мы утверждаем, что
	\begin{equation}\label{eq:H}
		|H| \le 2^s \cdot 2^{n-s-1-\varepsilon}\,.
	\end{equation}
	Действительно, согласно Лемме~\ref{lemma:isolatedweight} для каждого $x \in H$, $w'_F(x)=w_{F_j}(x)$ и для некоторого $j \in [2^s]$ $F_j$ не может принимать больше чем
	$2^{n-s-1-\varepsilon}$ изолированных решений веса хотя 
	$s+1+\varepsilon$. 
	
	Теперь мы покажем, что  
	\begin{equation}\label{eq:L}
		|L| \le m \cdot 2^{n}2^{\frac{-d}{s+1+\varepsilon}}(1/(s+1+\varepsilon))
	\end{equation}
	
	
	Пусть $F=\{C_1, \dotsc, C_m\}$. Для каждого $k \in [m]$, пусть $C'_k \subseteq C_k$
	будет дизъюнктом $C_k$, из которого удалены все недетерминированные переменные. Следовательно, для каждого
	$j \in [2^s]$, $F_j \subseteq \{C_1', \dotsc, C_m'\}$. Для $l \in [n]$,
	пусть $m_l=|\{k \in [m] \colon |C'_k|=l\}|$ будет количеством таких дизъюнктов длины $l$.
	Рассмотрим дизъюнкт $C'_k$
	и пусть $l=|C'_k|$. Тогда существует не более $l2^{n-l}$ пар $(x,i)$,
	где $x \in S$ и $i \in [n]$, таких что $C'_{F,x,i}=C_k'$:
	есть не более $l$ вариантов для $i$, фиксирование $i$ определяет значения
	всех $l$ литералов в $C_k'$ (все они равны нулю, кроме $i$-го), и существует не более $2^{n-l}$ вариантов для остальных битов $x$.
	Напомним, что $N_{l,F}(x)$ — это количество критических дизъюнктов относительно $x$
	длины $l$. Таким образом, мы приходим к следующему неравенству:
	\[m_l \cdot l \cdot 2^{n - l} \ge \sum_{x \in S} N_{F,l}(x) \ge \sum_{x \in L} N_{F,l}(x) \, .\]
	Тогда,
	\begin{equation}\label{eq:cc}
		m =\sum_{l \in [n]}m_l \ge \sum_{l \in [n]}\frac{\sum_{x \in L} N_{F,l}(x)}{l2^{n-l}}=
		\sum_{x \in L} \sum_{l \in [n]} \frac{N_{F,l}(x)}{l2^{n-l}} =
		\sum_{x \in L} d2^{-n} \sum_{l \in [n]} \frac{N_{F,l}(x)}{d} \cdot \frac{2^l}{l} \, .
	\end{equation}
	
	
	Для оценки последней суммы, пусть
	\[T(x)=\sum_{l \in [n]} \frac{N_{F,l}(x)}{d} \cdot \frac{2^l}{l}=\sum_{l \in [n]} \frac{N_{F,l}(x)}{d}\cdot g(l)\,,\]
	где $g(l) = \frac{2^l}{l}$. Так как $g(l)$ является выпуклой (для $l>0$) и $\sum_{l \in [n]}\frac{N_{F,l}(x)}{d}=1$, неравенство Йенсена дает
	\begin{equation}\label{eq:aa}
		T(x) \ge g\left(\sum_{l \in [n]} \frac{N_{F,l}(x)}{d}\cdot l\right) \, .
	\end{equation}
	Далее, неравенство Седракяна\footnote{Неравенство Седракяна является частным случаем неравенства Коши-Шварца: для всех $a_1, \dotsc, a_n \in \mathbb R$ и $b_1, \dotsc, b_n \in \mathbb{R}_{>0}$, $\sum_{i=1}^n a_i^2/b_i \ge \left(\sum_{i=1}^n a_i\right)^2/\sum_{i=1}^n b_i$.} (в сочетании с \eqref{eq:weight} и $\sum_{l \in [n]}N_{F,l}(x)=d$) дает
	\begin{equation}\label{eq:bb}
		\sum_{l \in [n]} l N_{F,l}(x) = \sum_{l \in [n]} \frac{N_{F,l}^2(x)}{N_{F,l}(x) / l} \ge \frac{(\sum_{l \in [n]} N_{F,l}(x))^2}{\sum_{l \in [n]} N_{F,l}(x) / l} = \frac{d^2}{w'_F(x)} \, .
	\end{equation}
	Так как $g(l)$ монотонно возрастает для $l \ge 1/\log 2$ и $w'_F(x)<s+1+\varepsilon$ для каждого $x \in L$,
	объединяя \eqref{eq:aa} и \eqref{eq:bb}, получаем
	\begin{equation}\label{eq:tx}
		T(x) \ge g \left( \frac{d}{w'_F(x)}\right) \ge g \left( \frac{d}{s+1+\varepsilon}\right) \, .
	\end{equation}
	Последнее неравенство верно, так как $\varepsilon \le d \log 2 - s - 1$. 
	
	Таким образом,
	\begin{align}
		m &\ge \sum_{x \in L}d2^{-n}T(x) \tag{\ref{eq:cc} и \ref{eq:tx}}\\
		& \ge \sum_{x \in L}d2^{-n}g \left( \frac{d}{s+1+\varepsilon}\right) \tag{определение $g$}\\
		&=|L|2^{-n}2^{\frac{d}{s+1+\varepsilon}}(s+1+\varepsilon)\,.
	\end{align}
	
	Используя \eqref{eq:H}, \eqref{eq:L} и факт, что $|H| + |L| = |S|$, получаем
	\begin{align}
		m &\ge (|S| - |H|) 2^{-n}2^{\frac{d}{s+1+\varepsilon}}(s+1+\varepsilon) \\
		&\ge (|S| - 2^{n-1-\varepsilon}) 2^{-n}2^{\frac{d}{s+1+\varepsilon}}(s+1+\varepsilon) \\
		&= (s+1+\varepsilon) 2^{\frac{d}{s+1+\varepsilon}} (|S|2^{-n} - 2^{-1-\varepsilon}).
	\end{align}
	
	\end{proof}


\subsection{Нижняя оценка для функции Parity}

В этом разделе мы доказываем, что верхняя оценка~\eqref{eq:upperm} на~$m$, показанная в~Разделе~\ref{sec:encodings}, является почти оптимальной.

\begin{theorem}\label{thm:main}
	Пусть $F$~— КНФ кодировка $\PAR_n$ с $m$ дизъюнктами и $s$ недетерминированными переменными. Тогда параметры $s$ и~$m$ не~могут быть одновременно слишком малы: если $s=O(n)$, то
	\begin{equation}\label{eq:sm}
		m \ge \Omega\left(\frac{s+1}{n}\right) \cdot 2^{\frac{n}{s+1}} \, .
	\end{equation}
\end{theorem}

Это неравенство является прямым следствием Леммы~\ref{lemma:nofclauses}.

\begin{proof}[Доказательство~\eqref{eq:sm}, $m \ge \Omega((s+1)2^{n/(s+1)}/n)$]
	Рассмотрим два случая.
	\begin{enumerate}
		\item $s \le n/2$.
		Пусть $S$~— множество $n$-изолированных решений $\PAR_n$. Заметим, что $|S| = 2^{n - 1}$. 
		По Лемме~\ref{lemma:nofclauses}, если 
		\begin{equation}\label{eq:ee}
			0 < \varepsilon \le n \ln 2 - s - 1\,,
		\end{equation}
		то
		\[ m \ge (s+1+\varepsilon) 2^{\frac{n}{s+1+\varepsilon}} \left(\frac{1}{2} - 2^{-1-\varepsilon}\right) = (s+1+\varepsilon) 2^{\frac{n}{s+1}} 2^{\frac{-n\varepsilon}{(s+1)(s+1+\varepsilon)}} \left(\frac{1}{2} - 2^{-1-\varepsilon}\right) \, .\]
		Установим $\varepsilon=1/n$ (неравенства~\eqref{eq:ee} выполняются, поскольку $s \le n/2$). Тогда,
		\[\left(\frac{1}{2} - \frac{1}{2^{\frac{1}{n} + 1}}\right) = \Theta\left(\frac{1}{n}\right)\,.\]
		Также,
		\[\frac{1}{2} \le 2^{\frac{-1}{(s+1)(s+1+1/n)}} \le 1\,,\]
		поскольку $2^{-1/x}$ возрастает для $x>0$.
		Таким образом,
		\[m \ge \Omega\left(\frac{s+1}{n} \cdot 2^{\frac{n}{s+1}}\right)\, .\]
		
		\item $n/2 < s =O(n)$. 
		В этом случае, нижняя оценка становится очевидной.
	\end{enumerate}
\end{proof}


\subsection{Нижняя оценка для функции Majority}

\begin{theorem}
	Пусть $F$~— КНФ кодировка $\MAJ_n$ с $m$~дизъюнктами и $s = O(n)$~недетерминированными переменными. Тогда параметры $s$~и~$m$ не~могут быть слишком малы одновременно:
	\begin{equation}
		m \ge \Omega\left(\frac{s+1 + \log n}{\sqrt{n}} \cdot 2^{\frac{n}{2(s+1 + \log n)}} \right) \, .
	\end{equation}
\end{theorem}

\begin{proof}
	Рассмотрим два случая.
	\begin{enumerate}
		\item $s \le n/2$.
		Пусть $S = \{x : \sum_{i = 1}^{n} x_i = \lceil n/2\rceil\}$. 
		Заметим, что $S \subseteq \MAJ_n^{-1}(1)$, и 
		\[|S| = { \binom{n}{\lceil n/2 \rceil}} \ge \frac{2^n}{\sqrt{n}}\, .\]
		По Лемме~\ref{lemma:nofclauses}, если 
		\begin{equation}\label{eq:ee2}
			\varepsilon \le \frac{n}{2} \ln 2 -s - 1,
		\end{equation}
		то
		\[ m \ge (s+1+\varepsilon) 2^{\frac{n/2}{s+1+\varepsilon}} \left(\frac{1}{\sqrt{n}} - 2^{-1-\varepsilon}\right).\]
		
		Установим $\varepsilon = \frac{1}{2}\log n$ (неравенства \ref{eq:ee2} выполняются, поскольку $s \le n/2$). Тогда,
		\begin{align}
			\left(\frac{1}{\sqrt{n}} - 2^{-1 - 1/2 \log n}\right) &= 
			\left(2^{- 1/2 \log n} - 2^{-1/2 \log n}/2 \right)\\ &= 
			2^{- 1/2 \log n} / 2 = \frac{1}{2\sqrt{n}} = 
			\Theta\left(\frac{1}{\sqrt{n}}\right)\,.
		\end{align}
		
		Следовательно,
		\[m \ge \Omega \left(\frac{s+1 + \log n}{\sqrt{n}} \cdot 2^{\frac{n}{2(s+1 + \log n)}} \right)\, .\]
		\item $n/2 < s = O(n)$.
		В этом случае мы должны показать, что $m \ge \Omega(\sqrt{n})$.
		Действительно, количество дизъюнктов должно быть хотя бы $\frac{n}{2}$, 
		иначе мы смогли бы выполнить формулу, присвоив менее $\frac{n}{2}$ переменных.
	\end{enumerate}
\end{proof}

