\specialsection{Заключение}

В ходе данной работы был проведен всесторонний анализ КНФ-кодировок симметрических функций, таких как функции четности (Parity) и голосования (Majority). Основное внимание было уделено исследованию нижних оценок на количество дизъюнктов в КНФ-кодировках и связи этих оценок с нижними оценками для булевых схем глубины 3.

Результаты показали, что для функции четности минимальное количество дизъюнктов в КНФ-кодировке зависит от числа недетерминированных переменных. В частности, если число недетерминированных переменных \( s \) пропорционально \( n \), то количество дизъюнктов \( m \) должно быть не менее \(\Omega((s+1) \cdot 2^{n/(s+1)}/n)\). Это демонстрирует, что параметры \( s \) и \( m \) не могут быть одновременно слишком малы. Для функции голосования была получена аналогичная оценка: если $s = O(n)$, то количество дизъюнктов \( m \) должно быть не менее \(\Omega((s+1 + \log n) \cdot 2^{n/(2(s+1 + \log n))}/\sqrt{n})\).

Одним из ключевых аспектов проведенного анализа является использование изолированных решений для доказательства нижних оценок. Было показано, что КНФ-кодировки функции четности и голосования содержат множество изолированных решений, что ограничивает возможности уменьшения числа дизъюнктов без увеличения числа недетерминированных переменных. Также был рассмотрен метод преобразования булевых схем в КНФ-кодировки, который позволил установить связь между нижними оценками для КНФ-кодировок и нижними оценками для схем глубины 3.

Несмотря на достигнутые результаты, данное исследование оставляет множество открытых вопросов и направлений для будущих работ. Одной из важнейших задач является улучшение существующих нижних оценок для КНФ-кодировок других симметрических функций и изучение их свойств. Более точные оценки могут привести к лучшему пониманию сложности булевых функций и эффективности их решения с помощью SAT-солверов.

Дополнительно, следует рассмотреть возможность разработки новых методов кодирования, которые могли бы уменьшить число дизъюнктов при сохранении приемлемого числа недетерминированных переменных. Такие методы могут быть особенно полезны в практических приложениях, где производительность SAT-солверов критически зависит от формы КНФ-кодировок.

Также перспективным направлением является исследование связи между КНФ-кодировками и другими классами булевых схем. Например, можно изучить возможность обобщения результатов на схемы с более сложными элементами или большими глубинами. Это позволит расширить область применения теоретических результатов на практические задачи в различных областях, таких как компьютерная томография, оптимизация радиочастотных назначений и конфигурация продуктов.

В заключение, дальнейшие исследования в данной области могут существенно способствовать развитию теории сложности вычислений и улучшению алгоритмических подходов к решению комбинаторных задач.